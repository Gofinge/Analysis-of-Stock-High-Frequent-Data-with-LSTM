\documentclass[hyperref]{article}
    \usepackage{graphicx} %% Package for Figure
    \usepackage{float} %% Package for Float
    \usepackage{amssymb}
    \usepackage{amsmath}
    \usepackage{mathtools}
    \usepackage[thmmarks,amsmath]{ntheorem} %% If amsmath is applied, then amsma is necessary
    \usepackage{bm} %% Bold Mathematical Symbols
    \usepackage{extarrows}
    \usepackage[hang,flushmargin]{footmisc} %% Let the footnote not indentation
    \usepackage[square,comma,sort&compress,numbers]{natbib} %% Sort of References
    \usepackage{mathrsfs} %% Swash letter
    \usepackage[font=footnotesize,skip=0pt,textfont=rm,labelfont=rm]{caption,subcaption} 
    %% Format of Caption for Tab. and Fig.
    \usepackage{booktabs} %% tables with three lines
    \usepackage{tocloft}
    %+++++++++++++++ Proof etc. +++++++++++++++++++++++++%%
    {%% Environment of Proof
        \theoremstyle{nonumberplain}
        \theoremheaderfont{\bfseries}
        \theorembodyfont{\normalfont}
        \theoremsymbol{\mbox{$\Box$}}
        \newtheorem{proof}{Proof}
    }
    
    \usepackage{theorem}
    \newtheorem{theorem}{Theorem}[section]
    \newtheorem{lemma}{Lemma}[section]
    \newtheorem{definition}{Definition}[section]
    \newtheorem{assumption}{Assumption}[section]
    \newtheorem{example}{Example}[section]
    \newtheorem{corollary}{Corollary}[section]
    {%% Environment of Remark
        \theoremheaderfont{\bfseries}
        \theorembodyfont{\normalfont}
        \newtheorem{remark}{Remark}[section]
    }
    %\numberwithin{equation}{section} %% Number of Equation
    %++++++++++++++++++++++++++++++++ Page format ++++++++++++++++++++++++++%%
    \graphicspath{{figure/}}                                 %% Path of Figures
    \usepackage[a4paper]{geometry}                           %% Paper size
    \geometry{left=2.5cm,right=2.5cm,top=2.5cm,bottom=2.5cm} %% Margin
    \linespread{1.2}                                         %% Line Spread
    %MS%%%%%%%%%%%%%%%%%%%%%%%%%%%% End Format %%%%%%%%%%%%%%%%%%%%%%%%%%%%%%%%%%
    
    %MS%%%%%%%%%%%%%%%%%%%%%%%%%%%%%%%%%%%%%%%%%%%
    %MS                                         %%
    %MS        The Main Body begins here        %%
    %MS                                         %%
    %MS%%%%%%%%%%%%%%%%%%%%%%%%%%%%%%%%%%%%%%%%%%%
    
    %MS++++++++++++++++++++++++++++++ Title +++++++++++++++++++
    \begin{document}
    \title{\bf A prediction model of sharing-bikes} 
    \date{}
    \author{\sffamily Zheng Xie, Guanglai Xu, Xiangfei Yin, Yewei Yuan\\}
    \renewcommand{\thefootnote}{\fnsymbol{footnote}}
    \footnotetext[1]{Corresponding author. }
    \maketitle
    
    %MS+++++++++++++++++++++ Abstract +++++++++++++++++++++++++
    {\noindent\small{\bf Abstract:}
        In this paper, we construct a model based on neural network to predict the destinations 
        of sharing-bikes. Basically, the model is composed of two system, prediction system and 
        evaluation system. Prediction system provides several predictions of destination based 
        on some useful constructed features. Evaluation system is a neural network which will 
        judge each prediction given by prediction system. Finally, the predictions with highest
        scores will be adopted. 
         
    }
    
    \vspace{1ex}
    {\noindent\small{\bf Keywords:}
        Keywords1; Keywords2;...}
    %MS++++++++++++++++++++++++++++++ Main body ++++++++++++++++++++
    \section{Introduction}
    In general, we can figure out several different methods to predict the destination, which 
    is what prediction system will do in the model. Some important features are needed in these 
    method. For instance, one can assume that a bike user will head for some ‘crowded’ place , 
    so we need to define a criteria to measure how ‘crowded’ a place is. Also, one can assume 
    that one will return where he start in his last ride. Different methods may lead to 
    different results, and it’s not easy to determine which method is the best for a certain 
    record. So we are trying to teach our model to distinguish good predictions from bad 
    predictions for each record. Here we use a simple neural network with only 1 hidden layer. 
    When the training of neural network, which we called evaluation system in this model, is 
    finished, we can use it to evaluate each prediction and select the one with highest score 
    to be the best prediction we desired.

    \section{Data analysis and preprocess}
    The name of each column of raw data are as follows: order_id, user_id, bike_id, bike_type,
    start_time, geohashed_start_location, geohashed_end_location. Geohash is a module which can
    hash location (in the form of latitute and longitute)
    
    \section{Construction of features}

    \section{Prediction methods}

    \section{Training of model}

    \section{Evaluation}

    \section*{Acknowledgments}
    %MS++++++++++++++++++++++++++++++ Reference ++++++++++++++++++
    \end{document}